The vegetation found in the Mediterranean region can thrive in two contrasting environments: areas with severe drought stress and mild winters, as well as colder regions, such as mountains, with extreme freezing temperature events in winter. These, even single events, can have a significant impact due to the daily temperature oscillations. This duality conditions the evolution of Mediterranean plant traits.~\cite{thompsonPlantTraitsEcological2020}\\

As a result of a temperature inversion, winter temperatures are several degrees lower in the St Martin-de-Londres basin (located 25 km north of Montpellier) compared to the surrounding hills. The distribution of different \textit{T. vulgaris} chemotypes reflect this drastic ecological gradient. There are two ecotypes adapted to this gradient, each comprising several chemotypes of monoterpenes.~\cite{thompsonPlantTraitsEcological2020,bataillonGenotypePhenotypeGenetic2022} \\

The ecological roles of monoterpenes are various, and the effects on the local soil environment and plant communities vary with the respective monoterpenes.~\cite{bataillonGenotypePhenotypeGenetic2022} The non-phenolic chemotypes (geraniol, $\alpha$-terpineol, thuyanol-4, and linalool) belong to the ecotype that dominates below 200 meters, where occasional winter frosts exclude the ecotype composed of phenolic chemotypes. Conversely, phenolic chemotypes (carvacrol and thymol) dominate above 250 meters because summer drought excludes the former ecotype.~\cite{thompsonPlantTraitsEcological2020}\\

However, during the last few years, extreme freezing events have drastically reduced due to, hypothetically, the ongoing climate warming. This fact presumably caused the observed rapid increase in the frequency of the phenolic ecotype in the area, which has been under study over the last 50 years.~\cite{thompsonPlantTraitsEcological2020,bataillonGenotypePhenotypeGenetic2022} \\

Previous studies have shown that genetic variation within genes of the monoterpene pathway explains a large fraction of chemotype variation.~\cite{bataillonGenotypePhenotypeGenetic2022} Understanding the genetics of this polymorphism is critical in understanding its effects on community dynamics. It can also provide insight into fundamental questions in genetics: How does a species maintain multiple adaptive traits even in the presence of gene flow? How do plant populations respond to a decrease in strong selection pressure?~\cite{bataillonGenotypePhenotypeGenetic2022, thompsonPlantTraitsEcological2020}\\

Resolving the genome structure of \textit{T. vulgaris} would be crucial to unlocking the genetics that determines its chemotypes and other traits. Improvements in third-generation or long-read technologies make genome sequencing faster and more affordable, and it is the current method of choice to generate highly contiguous plant genome assemblies.~\cite{puckerPlantGenomeSequence2022} \ac{HiFi} reads are long, 10–25 kb, and highly accurate, over 99.5\%.~\cite{honHighlyAccurateLongread2020} With \ac{HiFi} reads, we can obtain more accurate and complete assemblies without a correction step, as we need with noisy long reads.~\cite{puckerPlantGenomeSequence2022}\\

Among the algorithms available for \textit{de novo} assembly using long reads, Hifiasm, designed explicitly for \ac{HiFi} reads, stands out.~\cite{chengHaplotyperesolvedNovoAssembly2021} However, current algorithms can struggle to differentiate between genome repeats that approach the length of long reads, generating hundreds and thousands of incomplete contigs.~\cite{huangAlignGraph2SimilarGenomeassisted2021} \\

More complete scaffolds, ideally assembled at the chromosomal level, would benefit the study of \textit{T. vulgaris}. Among other things, we could determine linkage groups, estimate distances between \textit{loci}, and avoid the complexity of working with a fragmented genome of thousands of contigs.~\cite{tamazianChromosomerReferencebasedGenome2016}\\

Experimental approaches such as \ac{Hi-C} or optical mapping (BioNano) can facilitate contig joining.~\cite{jungToolsStrategiesLongRead2019} From \ac{Hi-C} data, we can create representations of complete chromosomes known as pseudo-chromosomes. These pseudo-chromosomes consist of ordered contigs linked by gaps (a sequence of ambiguous bases, Ns). Gaps only provide information on the distance between specific sequences. ~\cite{puckerPlantGenomeSequence2022}\\

Computational approaches benefit from non-requiring physical chromosome maps. Different algorithms in the literature try to solve this problem by using a closely related genome assembly. The reference provides approximate information for the target genome assembly that we can integrate to improve the quality of our assembly.~\cite{huangAlignGraph2SimilarGenomeassisted2021}\\

The recently published genome of \textit{T. quinquecostatus}, close-related to \textit{T. vulgaris}~\cite{sunPopulationDiversityAnalyses2023},  fits this purpose perfectly.~\cite{sunChromosomelevelAssemblyAnalysis2022}. \textit{T. quinquecostatus} reference genome sequence was obtained from \textit{de novo} assembly of  \ac{HiFi} reads and \ac{Hi-C} data.\\ 

Most of these approaches use a \textit{de novo} assembly as a starting point, except those so-called reference-guided assembly methods.~\cite{ReferenceguidedAssemblyFour,lischerReferenceguidedNovoAssembly2017} Roughly, some methods reassemble the contigs, sometimes referred to as similar genome-assisted reassembly methods~\cite{baoAlignGraphAlgorithmSecondary2014,huangAlignGraph2SimilarGenomeassisted2021,baoReMILOReferenceAssisted2018}, and others order and arrange contigs (from now on scaffold), sometimes called synteny-based or homology-based scaffolding methods.~\cite{kimReferenceassistedChromosomeAssembly2013,FenderglassRagoutChromosomelevel,alongeAutomatedAssemblyScaffolding2022}\\

RagTag is a suite of tools for genome assembly scaffolding.~\cite{FenderglassRagoutChromosomelevel} We can use it to perform a whole-genome alignment to the reference assembly and scaffolds the target assembly. This method has the advantage of not modifying the original contig sequences. It also joins the initial contigs with Ns. The resulting scaffolds are conceptual and practically equivalent to the pseudo-chromosomes obtained by \ac{Hi-C} data; hence, we will interchange the two terms. 