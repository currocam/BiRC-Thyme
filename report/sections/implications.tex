\chapter{Future work}\label{cha:future}

\subsection*{Quality evaluation}

Further research should perform a more comprehensive assessment of the obtained genome's quality. By performing a more exhaustive evaluation, we could rigorously address whether the scaffolds (pseudo-chromosomes) obtained can be considered "chromosome-level" and identify potential improvement areas.\\

Further evaluation of the accuracy and completeness of our scaffolded genome assembly is recommended for future work. Exploring options such as annotating sequence repeats using tools like RepeatModeler~\cite{flynnRepeatModeler2AutomatedGenomic2019} or identifying the exon-intron structure of conserved protein families~\cite{parraCEGMAPipelineAccurately2007} are options available. \\

Since the presence of repetitive sequences is one of the major challenges in plant genome assembly, and we have evidence that the same applies here, we should not limit ourselves to annotating repetitive sequences but assess the assembly of repeat space. One method by which to achieve this would be LAI (LTR Assembly Index), which evaluates assembly continuity using LTR retrotransposons.~\cite{ouAssessingGenomeAssembly2018}\\

\subsection*{Experimental validation}

More rigorous validation of the assembly should use additional experimental data to the Illumina sequences used in this project. While obtaining a contact map through \ac{Hi-C} would provide the most reliable validation, this approach is adopted precisely due to the absence of such data. \\

A possible line of work would be to align the available transcriptome of \textit{T. vulgaris}~\cite{bataillonGenotypePhenotypeGenetic2022} to the genome assembly. The information obtained would be helpful to corroborate the validity of the assembly and could also be used to annotate the predicted protein-coding genes.~\cite{sunChromosomelevelAssemblyAnalysis2022,haasImprovingArabidopsisGenome2003} \\

If Illumina reads from whole-genome sequencing were available, another option would be to align these reads to the assembly and study the coverage, paying particular attention to signs of incomplete areas. In addition, we could use population genetic data to gain insights into the potential impact of gaps by comparing experimental linkage with predicted chromosomal distances.\\

\subsection*{Further improvement of the assembly}

Future works could also focus on testing alternative reference scaffold algorithms. However, additional analysis is necessary before implementing more complex algorithms to evaluate if betters scaffolds can be obtained with available data. \\

Other algorithms like AlignGraph2 rely on the similarity between the \textit{de novo} assembly and the reference but do not preserve the original sequence.~\cite{huangAlignGraph2SimilarGenomeassisted2021} If that is the case, using a more complex and less traceable algorithm might lead to a false perception of accuracy. \\

%Our current approach prioritizes transparency and aims to obtain conservative results.\\  

\subsection*{Genomic analysis}

To gain insights into the evolutionary history of \textit{T. vulgaris}, we might conduct genomic analyses using the obtained genome assembly. For instance, we could extend the phylogenetic analysis of \textit{T. quinquecostatus} conducted by Sun \etal~\cite{sunChromosomelevelAssemblyAnalysis2022} by including the \textit{T. vulgaris} genome, which is not presently available, nor any other genome in the genus \textit{Thymus}. \\

The initial motivation for this project was the study of thyme ecotypes in St Martin-de-Londres basin. With this genome assembly, conducting an in-depth analysis of the  \textit{loci} identified by Bataillon \etal~\cite{bataillonGenotypePhenotypeGenetic2022} as potentially responsible for thyme ecotype genetic diversity will be possible.\\

This investigation can provide valuable insights into various aspects, including the relative chromosomal positioning of these \textit{loci}, and gain a deeper understanding of their functional implications and the evolutionary processes that are taking place. \\

\chapter{Conclussions}\label{cha:conclusion}


In this work, we have evaluated the effectiveness of \ac{HiFi} sequencing for the \textit{de novo} assembly of plant genomes and demonstrated the utility of homology-based scaffolding to reconstruct a fragmented assembly into chromosome-level scaffolds using computational methods. We have done so through a reproducible preprocessing pipeline and a transparent computational approach. \\

Our data consistently suggest that the de novo assembly we have generated is high quality, albeit very fragmented. Based on the N50, BUSCO analysis, cytometric estimates of the \textit{T. vulgaris} genome size and validation with independent experimental data, we have compelling evidence that our assembly represents a chromosome-level assembly to some extend.\\

Quality assessment and experimental validation can be significantly improved. A more extensive set of metrics is essential to obtain more conclusive answers to the research questions formulated for this work. Nevertheless, the results we have presented in this work are promising.\\

With proper consideration of current limitations and knowledge gaps, this assembly provides a valuable resource for future genomic studies, contributing to advancing knowledge in the field of \textit{T. vulgaris} ecology.\\
