The evolutionary ecology of Mediterranean thyme (\textit{Thymus vulgaris}) has been studied for more than 50 years regarding how variations in secondary compound production are adaptive in different climatic conditions.~\cite{thompsonPlantTraitsEcological2020}\\

In the South of France, the distribution of different chemical phenotypes (hereafter chemotypes) is consistent with differences in the abiotic environment. How are these chemotypes maintained within species in such a close geographic area, and which are the ecological genetics underlying this phenomenon are relevant questions in the field.~\cite{bataillonGenotypePhenotypeGenetic2022}\\

As with other non-model organisms, the absence of a sequenced and accurately annotated reference genome limits the scope of the possible population genomic studies. However, the advances in long-read sequencing technologies make plant genome sequencing more affordable and accessible.~\cite{puckerPlantGenomeSequence2022} \\

The primary motivation of this project is to obtain a chromosome-level assembly of \textit{T. vulgaris}. The \ac{HiFi} sequencing of one thyme genotype and a \textit{de novo} assembly made with the same reads was available before the project started, as well as a chromosome-level assembly of a related specie. Based on that, I chose the following research questions for my Project in Bioinformatics:

\begin{enumerate}
    \item Can a chromosome-level assembly be obtained from a \textit{T. vulgaris de novo} assembly generated using \ac{HiFi} reads by incorporating information from a closely related genome?
    \item Is a homology-based assembly scaffolding approach effective in the absence of \ac{Hi-C} data?
\end{enumerate}