The evolutionary ecology of Mediterranean thyme (\textit{Thymus vulgaris}) have been studied with for more than 50 years regarding how variations in secondary compound production are adaptive in different climatic conditions.~\cite{thompsonBibliography2020} In the South of France, the distribution of different chemical phenotypes (hereafter chemotypes) is consistent with differences in the abiotic environment. How are these chemotypes maintained within species in such a close geographic area, and which are the ecological genetics underlying this phenomenon are relevant questions in the field.~\cite{bataillonGenotypePhenotypeGenetic2022}\\

As with other non-model organisms, the absence of a sequenced and accurately annotated reference genome limits the scope of the possible population genomic studies. However, the advances in long-read sequencing technologies make plant genome sequencing more affordable and accessible.~\cite{puckerPlantGenomeSequence2022} \\

The primary motivation of this project is to obtain a chromosome-level assembly of \textit{T. vulgaris}. To achieve this, we made a de novo assembly using High-Fidelity (HiFi) PacBio long reads. We subsequently arranged the contigs we obtained into pseudochromosomes using a homology-based assembly scaffolding approach, using the closely-related \textit{T. quinquecostatus} genome as a reference. In this report, I will try to answer the questions:

\begin{enumerate}
    \item Is it possible to obtain a chromosome-level assembly from a  \textit{T. vulgaris de novo} assembly generated using High-Fidelity (HiFi) PacBio long reads by incorporating information from a closely related genome?
    \item Is an homology-based assembly scaffolding approach an effective strategy for this purpose?
\end{enumerate}

