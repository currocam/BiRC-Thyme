\section*{Quality Control of raw reads}

The computed theoretical coverage and the quality assessment of raw Hifi reads suggests they are high quality and appropriate for genome assembly. We have not discovered anything indicating otherwise among the different criteria utilized, and the length distribution aligns with the literature.~\cite{honHighlyAccurateLongread2020}\\

In plants, it is not uncommon to observe slightly multimodal GC\% distributions. GC content might exhibit an irregular or bimodal distribution among angiosperm plants, although most dicots, such as \textit{T. vulgaris}, tend to show more normal distributions.~\cite{bowersGCContentPlant2022} Further research could delve deeper into this phenomenon and explore the discrepancies with the unimodal GC content observed in the genome of \textit{T. quinquecostatus}.~\cite{sunChromosomelevelAssemblyAnalysis2022}\\

\section*{Exploratory analysis of long reads and reference genome}

We mapped the top five percent of \textit{T. vulgaris} long reads to the reference genome of \textit{T. quinquecostatus} during our data exploration. Most reads were successfully mapped, 97.6\%. Our results strongly suggest that a zero-inflated Poisson distribution is a sensible model for accurately capturing the distribution of reads across the genome of  \textit{T. quinquecostatus}.\\

We propose the following biological interpretation: \textit{T. quinquecostatus} genome consists of conserved and non-conserved regions with \textit{T. vulgaris}. Those non-conserved regions probably share some homology, but we cannot detect it. In the model, mapping reads to the conserved areas occurs according to a Poisson process, while the probability of mapping to non-conserved areas is assumed to be zero. By setting this probability to zero, we assume that the mutation rate between the two species is such that, without selective pressure, the DNA sequence has lost its original similarity.\\
 
Following the previous interpretation, the proportion of non-conserved regions in \textit{T. quinquecostatus} compared to \textit{T. vulgaris} remains relatively constant across chromosomes, ranging from 74\% to 77\%. The most likely factor is the high percentage of duplicated sequences in the \textit{T. quinquecostatus} genome,  71\%.~\cite{sunChromosomelevelAssemblyAnalysis2022} We do not have that data available for \textit{T. vulgaris} and the annotation of repeated sequences is beyond this project's scope. \\

Moreover, we used a subset of the best-quality reads in this analysis. Hence, it is plausible that technical biases have resulted in some areas of the \textit{T. vulgaris} genome being systematically sequenced with reads of shorter length or worse quality. The best five percent filtering process can be examined with the same reasoning.\\

Data exploration reveals that most sequences map to a relatively small fraction of each chromosome, indicating insufficient conservation to perform a reference-based assembly. However, it is suitable to generate contigs \textit{de novo} and subsequently use the fraction of the genome to which we successfully map reads to understand the contigs' genomic context better.\\ 

\section*{De novo assembly}

The \textit{de novo} assembly of the genome is highly fragmented, consisting of 1,884 contigs, likely due to the high prevalence of repeated sequences. Currently, there is no literature available that specifically addresses the expected level of fragmentation for plant genomes using different assembly algorithms.\\

However, for \textit{Vitis vinifera}, it has been demonstrated that current algorithms can assemble long reads into 591-1066 contigs.~\cite{huangAlignGraph2SimilarGenomeassisted2021} It is important to note that the estimated 1C value of \textit{V. vinifera} is 392Mb,~\cite{PlantDNACvalues} which is roughly half the size of the \textit{T. vulgaris} genome. Considering this, the number of contigs in our assembly falls within the expected range.\\

Additional research is required to examine why there is a difference in size between the assembly and the cytometric estimation. The first step should involve assessing how much this difference can be attributed to experimental error and performing estimations with alternative methods, such as k-mer counting.~\cite{pflugMeasuringGenomeSizes2020} Other possible factors, such as assembly artifacts or structural variations, should be explored. \\

\section*{Homology-based assembly scaffolding}

The 13 scaffolds that we hypothesized to be equivalent to \textit{T. quinquecostatus} pseudo-chromosomes account for 77\% of the assembly size. The reference assembly of \textit{T. quinquecostatus} have 8\% of unplaced contigs.~\cite{sunChromosomelevelAssemblyAnalysis2022} Therefore, expecting a lower value of unplaced contigs in our assembly would be unrealistic.\\

The estimated size of the 13 pseudo-chromosomes in this study corresponds to approximately 92.16\% of the haploid size of \textit{T. vulgaris}, as determined through cytometric analysis. This suggests that the 13 scaffolds identified in our study likely represent the actual chromosomes of \textit{T. vulgaris} to a significant extent.\\

Nevertheless, the assembly has a high number of gaps, 819. By comparison, the reference genome used, based on Hi-C, has 150 gaps. Addressing the impact of these gaps is not trivial. It is, in fact, a common weakness in assemblies related to the existence of chromosomal structures, such as centromeres, which are more likely to have repeated sequences.~\cite{peonaHowCompleteAre2018} \\

Even if we do not know the sequence, knowing the approximate size of the gaps would enable us to calculate chromosomal distances with greater precision. Our method uses the whole genome alignment between \textit{T. vulgaris} and \textit{T. quinquecostatus} to infer gap size. By doing this, we can achieve a more refined representation of the genome by assuming a certain level of conservation between the two species. However, we only have conclusive results from a small fraction of the gaps, 8.3\%. \\



\section*{Assembly quality assessment}

Regarding the quality assessment, our analysis was limited to the BUSCO analysis and examining the contiguity using N50. A more comprehensive evaluation is necessary to determine the accuracy and completeness of the assembly. In particular, further investigations should include identifying repetitive elements and functional annotation.\\

The scaffold N50 value in our assembly is 48.92 Mb, which is 26-fold higher than the contig N50. This is an exceptionally high number compared with the N50 statistics of plant genomes published in the last decade.~\cite{sunTwentyYearsPlant2022a}\\

The BUSCO's assembly completeness score is satisfactory, but it is worth noting the significantly higher percentage of complete and duplicated BUSCO genes in \textit{T. vulgaris} compared to \textit{T. quinquecostatus}. This discrepancy could be due to technical factors, such as a chimeric assembly of haplotypes. Alternatively, a recent duplication event in the evolutionary history of \textit{T. vulgaris} may explain the disparity and account for the genome size difference.~\cite{manniBUSCOAssessingGenomic2021}\\

It is essential to acknowledge that neither of these metrics provides an unbiased assessment of genome completeness. Firstly, incorrectly assembled contigs will inflate the N50 value. Secondly, a high BUSCO score does not necessarily guarantee genome completeness. Many recently assembled or draft genomes exhibit high BUSCO scores indistinctly. A comprehensive of different tools is crucial to evaluate the assembly quality.~\cite{mokhtarLargescaleAssessmentQuality2023} \\

%Therefore, we expect the most reliable chromosomal distances to be between \textit{loci} at the extremes. \\

\section*{Mapping short Illumina reads to scaffolded assembly}

%The short Illumina reads were sequenced after a hybridization protocol, in which baits hybridize to some areas of interest, then enrich those same regions. \\

The results of aligning sequences obtained independently from different individuals of \textit{T. vulgaris} and \textit{S. montana} strongly support the validity of our assembly (see \autoref{fig:error_rate}).\\

The \textit{T. vulgaris} sequences align more accurately with fewer mismatches to the \textit{T. vulgaris} genome than the \textit{S. Montana} sequences. No substantial differences in the percentage of aligned sequences were observed between different \textit{T. vulgaris} individuals. However, as expected, the lowest number of mismatches occurred when using sequences obtained from the same individual to construct the reference genome, reflecting some degree of intraspecies genetic variation.\\

A more stringent hybridization condition, hybridization 2, slightly increased the percentage of mapped \textit{T. vulgaris} sequences and the number of mismatches. It may indicate that hybridization 2 enriches the number of reads corresponding to regions with homology to the \textit{T. vulgaris} region of interest but, at the same time, increases mismatches due to interspecies genetic variation.\\

Although these experiments support what we expected from experimental conditions,~\cite{bataillonGenotypePhenotypeGenetic2022} experiments with a larger number of subjects would have to be performed in order to be able to make statistical statements, which is beyond this project's scope.